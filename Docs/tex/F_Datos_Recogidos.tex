\apendice{Datos recogidos}


\section{Introducción}
En esta sección se ponen en comparativa algunos de los resultados obtenidos. Estos resultados están extraídos de un conjunto no muy grande de datos que tuve que generar yo mismo. La intención era que muchos usuarios jugasen y me entregasen los datos,pero esto no ha sido posible debido a la continua mutación del proyecto. Los datos tomados una semana a la siguiente ya no servían. 

\section{Árboles de decisión}

En entrenamiento con los árboles de decisión resultó, en cierto modo, desesperanzador. El primer entrenamiento que se realizó, utilizando un random forest, resultó desastroso. En la tabla que se muestra a continuación se ha entrenado un random forest con $9000$ muestras y se ha ejecutado cinco veces. Cada una de las ejecuciones conlleva diez rondas.

% Please add the following required packages to your document preamble:
% \usepackage[table,xcdraw]{xcolor}
% If you use beamer only pass "xcolor=table" option, i.e. \documentclass[xcolor=table]{beamer}
\begin{table}[h!]
\centering
\begin{tabular}{|l|l|l|l|l|l|l|l|l|l|l|}
\hline
\rowcolor[HTML]{C0C0C0} 
parrtida/ronda & 1   & 2    & 3    & 4     & 5    & 6     & 7   & 8     & 9     & 10   \\ \hline
1              & 278 & -293 & 45   & -268  & -213 & 273   & 36  & 103   & -52   & 257  \\ \hline
\rowcolor[HTML]{EFEFEF} 
2              & 164 & 419  & 1461 & 48    & 517  & 301   & -3  & 86    & -79   & 512  \\ \hline
3              & 500 & 148  & 376  & -84   & 3    & -1328 & 44  & 489   & -1763 & 119  \\ \hline
\rowcolor[HTML]{EFEFEF} 
4              & 222 & 5    & 220  & 179   & 878  & 113   & -92 & -1046 & -72   & 95   \\ \hline
5              & -57 & 520  & 157  & -2311 & -72  & -395  & 961 & 398   & -72   & -150 \\ \hline
\end{tabular}
\caption{Entrenamiento con Random forest}
\label{tab:RndForest}
\end{table}

Como se puede observar en la tabla \ref{tab:RndForest}, los resultados son muy malos, tan pronto puede alcanza mas de mil puntos como acabar hundido en los valores negativos. Cuando se observaron este tipo de resultados por primera vez, se pensó que el problema podría residir en un sobre-ajuste de la función. Esto me condujo a probar un modelo más simple, un DecissionClassifier (árbol de decisión). Este modelo, al ser algo mas simple, no se ajusta tanto.


% Please add the following required packages to your document preamble:
% \usepackage[table,xcdraw]{xcolor}
% If you use beamer only pass "xcolor=table" option, i.e. \documentclass[xcolor=table]{beamer}
\begin{table}[h!]
\centering
\begin{tabular}{|l|l|l|l|l|l|l|l|l|l|l|}
\hline
\rowcolor[HTML]{C0C0C0} 
parrtida/ronda & 1   & 2   & 3   & 4    & 5    & 6    & 7   & 8    & 9    & 10  \\ \hline
1              & 617 & 67  & -6  & 95   & 109  & 261  & 151 & -257 & -76  & 253 \\ \hline
\rowcolor[HTML]{EFEFEF} 
2              & 217 & 66  & -34 & 120  & -174 & 318  & 200 & 306  & -340 & 897 \\ \hline
3              & 189 & 11  & 180 & 44   & 320  & 146  & 55  & 369  & 283  & 273 \\ \hline
\rowcolor[HTML]{EFEFEF} 
4              & 173 & 33  & 503 & -394 & 47   & -158 & 189 & 131  & 320  & -72 \\ \hline
5              & 575 & 426 & 149 & -11  & 358  & 76   & 61  & 110  & 739  & 639 \\ \hline
\end{tabular}
\caption{Entrenamiendo con DecissionTree}
\label{my-label}
\end{table}



Estos resultados ta eran algo mejores, el ya se mueve, en cierto modo, de forma «inteligente», aunque los resultados no son los mejores. Una vez comprobado que había esperanzas de conseguir un aprendizaje, se pasó directamente a los algoritmos evolutivos.


\section{Algoritmos evolutivos}

Para el primer entrenamiento con redes neuronales se emplea un perceptrón multicapa. El algoritmo utilizado es el \emph{eaSimple}.

% Please add the following required packages to your document preamble:
% \usepackage[table,xcdraw]{xcolor}
% If you use beamer only pass "xcolor=table" option, i.e. \documentclass[xcolor=table]{beamer}
\begin{table}[]
\centering
\begin{tabular}{|l|l|l|l|l|l|}
\hline
\rowcolor[HTML]{EFEFEF} 
gen & nevals & avg      & std     & min     & max    \\ \hline
0   & 10     & -738.945 & 1093.75 & -2913.4 & 706.7  \\ \hline
1   & 10     & 464.033  & 397.032 & -394    & 1043.6 \\ \hline
2   & 8      & 576.64   & 332.162 & -67.5   & 1048   \\ \hline
3   & 6      & 590.65   & 434.868 & -549    & 1048   \\ \hline
4   & 10     & 323.44   & 543.768 & -573.7  & 1180.3 \\ \hline
5   & 4      & 849.94   & 280.792 & 510.1   & 1180.3 \\ \hline
6   & 6      & 729.28   & 580.734 & -698.4  & 1263.5 \\ \hline
7   & 8      & 783.804  & 266.751 & 479.909 & 1263.5 \\ \hline
8   & 5      & 712.796  & 544.912 & -312.7  & 1263.5 \\ \hline
9   & 8      & 604.24   & 549.538 & -326    & 1263.5 \\ \hline
10  & 4      & 1082.16  & 285.419 & 579.091 & 1294.4 \\ \hline
\end{tabular}
\caption{Entrenamiento con algoritmos evolutivos 1}
\label{tab:entr1}
\end{table}



En este segundo entrenamiento se ha partido como invididuo inicial el mejor del anterior entrenamiento. El modelo de algoritmo evolutivo es igual que el anterior.

% Please add the following required packages to your document preamble:
% \usepackage[table,xcdraw]{xcolor}
% If you use beamer only pass "xcolor=table" option, i.e. \documentclass[xcolor=table]{beamer}
\begin{table}[]
\centering
\begin{tabular}{|l|l|l|l|l|l|}
\hline
\rowcolor[HTML]{EFEFEF} 
gen & nevals & avg      & std     & min     & max                              \\ \hline
0   & 10     & -222.475 & 1515.29 & -3194.5 & 2281.1                           \\ \hline
1   & 10     & 1292.45  & 1194.07 & -1305.2 & 2830.4                           \\ \hline
2   & 8      & 2094.54  & 776.345 & 172.8   & 3143.55                          \\ \hline
3   & 6      & 2432.41  & 1110.06 & -558.8  & 3650.6                           \\ \hline
4   & 10     & 1653.41  & 1149.34 & -515.3  & 3162                             \\ \hline
5   & 4      & 2733.52  & 435.733 & 1907.9  & 3250.9                           \\ \hline
6   & 6      & 2502.06  & 952.354 & -80     & 3250.9                           \\ \hline
7   & 8      & 2429.7   & 509.653 & 1850.7  & 3250.9                           \\ \hline
8   & 5      & 2694.2   & 781.888 & 614.5   & 3250.9                           \\ \hline
9   & 8      & 1929.75  & 1149.92 & -539.7  & 3250.9                           \\ \hline
10  & 4      & 2952.38  & 515.971 & 1728.09 & 3250.9 
\\ \hline
\end{tabular}
\caption{Entrenamiento algoritmos evolutivos 2}
\label{tab:entr2}
\end{table}



Haciendo el análisis de datos me percaté de que mi función de \emph{fitness} no era todo lo buena que yo esperaba. Al analizarlo con detenimiento me di cuenta de que no me interesa tener un individuo con mucha suerte, me interesa un individuo que siempre juega bien. Hasta el momento, toda la evolución ha tenido como meta maximizar la media de las partidas que juega un individuo, una función de evaluación a primera vista válida, pero ¿qué ocurre si dos individuos muy parecidos juegas sendas partidas, y uno es más afortunado que el otro en una única ronda? La media de este individuo sera ligeramente mayor y será marcado como más apto, y todo por azar. llegado a este punto pensé que yo no quiero saber quién tiene mas suerte, sino quién juega de una forma más regular. A partir de ahora, le voy a aplicar una penalización en función de la desviación estándar, con la esperanza de obtener como resultado individuos mas regulares.

% Please add the following required packages to your document preamble:
% \usepackage[table,xcdraw]{xcolor}
% If you use beamer only pass "xcolor=table" option, i.e. \documentclass[xcolor=table]{beamer}
\begin{table}[h!]
\centering
\begin{tabular}{llllll}
\hline
\rowcolor[HTML]{EFEFEF} 
\multicolumn{1}{|l|}{\cellcolor[HTML]{EFEFEF}gen} & \multicolumn{1}{l|}{\cellcolor[HTML]{EFEFEF}nevals} & \multicolumn{1}{l|}{\cellcolor[HTML]{EFEFEF}avg} & \multicolumn{1}{l|}{\cellcolor[HTML]{EFEFEF}std} & \multicolumn{1}{l|}{\cellcolor[HTML]{EFEFEF}min} & \multicolumn{1}{l|}{\cellcolor[HTML]{EFEFEF}max} \\ \hline
\multicolumn{1}{|l|}{0}                           & \multicolumn{1}{l|}{10}                             & \multicolumn{1}{l|}{-0.903205}                   & \multicolumn{1}{l|}{2.50895}                     & \multicolumn{1}{l|}{-6.84849}                    & \multicolumn{1}{l|}{2.64933}                     \\ \hline
\multicolumn{1}{|l|}{1}                           & \multicolumn{1}{l|}{10}                             & \multicolumn{1}{l|}{1.11837}                     & \multicolumn{1}{l|}{0.875644}                    & \multicolumn{1}{l|}{-1.02659}                    & \multicolumn{1}{l|}{2.4963}                      \\ \hline
\multicolumn{1}{|l|}{2}                           & \multicolumn{1}{l|}{8}                              & \multicolumn{1}{l|}{1.65532}                     & \multicolumn{1}{l|}{0.904609}                    & \multicolumn{1}{l|}{-0.202067}                   & \multicolumn{1}{l|}{3.68612}                     \\ \hline
\multicolumn{1}{|l|}{3}                           & \multicolumn{1}{l|}{6}                              & \multicolumn{1}{l|}{2.40992}                     & \multicolumn{1}{l|}{1.3058}                      & \multicolumn{1}{l|}{-0.600178}                   & \multicolumn{1}{l|}{3.68612}                     \\ \hline
\multicolumn{1}{|l|}{4}                           & \multicolumn{1}{l|}{10}                             & \multicolumn{1}{l|}{1.4735}                      & \multicolumn{1}{l|}{1.14359}                     & \multicolumn{1}{l|}{-0.479148}                   & \multicolumn{1}{l|}{3.36271}                     \\ \hline
\multicolumn{1}{|l|}{5}                           & \multicolumn{1}{l|}{4}                              & \multicolumn{1}{l|}{2.76411}                     & \multicolumn{1}{l|}{0.548491}                    & \multicolumn{1}{l|}{1.93125}                     & \multicolumn{1}{l|}{3.48607}                     \\ \hline
\multicolumn{1}{|l|}{6}                           & \multicolumn{1}{l|}{6}                              & \multicolumn{1}{l|}{2.34142}                     & \multicolumn{1}{l|}{1.21775}                     & \multicolumn{1}{l|}{0.115697}                    & \multicolumn{1}{l|}{3.48607}                     \\ \hline
\multicolumn{1}{|l|}{7}                           & \multicolumn{1}{l|}{8}                              & \multicolumn{1}{l|}{2.43485}                     & \multicolumn{1}{l|}{0.819595}                    & \multicolumn{1}{l|}{1.40789}                     & \multicolumn{1}{l|}{3.82011}                     \\ \hline
                                                  &                                                     &                                                  &                                                  &                                                  &                                                 
\end{tabular}
\caption{Entrenamiento con nuevo fitness}
\label{tab:nuevoFitness}
\end{table}

