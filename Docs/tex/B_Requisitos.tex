\apendice{Especificación de Requisitos}

\section{Introducción}
En este apartado se especificarán las necesidades funcionales que se pretende alcanzar. Para ello se describirán los requisitos del software que se quiere desarrollar.

\section{Objetivos generales}
\begin{itemize}
    \item Como ya de describió en la memoria, el objetivo del proyecto es diseñar y desarrollar un videojuego de tipo «arcade». Para la implementación del juego que sirve como base para este proyecto, es necesario planear las mecánicas de juego, además de diseñar los modelos del jugador, los enemigos y otros elementos gráficos. Esta tarea también incluye el diseño de la interfaz, banda sonora y efectos de sonido. 
    
    \item Diseñar e implementar un sistema inteligente que posteriormente será utilizado para interactuar con el videojuego creado. Este proceso de divide en:
    \begin{itemize}
        \item Establecer una comunicación entre el juego y el modelo del sistema inteligente. Dado que el juego y el script que hace uso del modelo serán implementados en diferente lenguaje de programación, se debe establecer un mecanismo de intercambio de datos para la comunicación.
        \item Diseñar el conjunto de datos que posteriormente servirá para el entrenamiento del modelo. Al ser el punto de partida el aprendizaje por imitación, necesitaremos un gran volumen de datos de los cuales seleccionaremos todos o solo los más representativos para el entrenamiento.
        \item Implementar, haciendo uso de las bibliotecas como scikit-learn (minería de datos), DEAP(algoritmos evolutivos) y Pandas y NumPy (procesamiento de datos a bajo nivel) el sistema inteligente que jugará al videojuego.
    \end{itemize}
\end{itemize}

\section{Catalogo de requisitos}

\subsection{Especificación de requisitos}
A continuación se procede a describir los requisitos \emph{iniciales} del proyecto. Hay que recalcar que a lo largo del desarrollo se fueron descubriendo carencias que tenían que cubrir nuevas nececidades. Estos nuevos requisitos funcionales no se describen en este apartado, pero pueden ser consultados el la memoria, en el apartado de «Aspectos relevantes».

\subsection{Requisitos funcionales}

\begin{itemize}
    \item RF-1: El juego debe ser de tipo arcade en 2D. Un juego infinito en el que el jugador aspira a alcanzar la máxima puntuación de forma individual. 
    \item RF-2: Controlamos una nave espacial.
        \begin{itemize}
            \item RF-2.1: El jugador tiene una única vida.
            \item RF-2.2: El movimiento del jugador es horizontal.
            \item RF-2.3: El jugador se mueve a velocidad constante.
            \item RF-2.4: El jugador siempre dispara hacia delante en línea recta.
        \end{itemize}
    \item RF-3: El jugador ha de eliminar el máximo número de enemigos posibles sin impactar contra ellos
    \item RF-4: Dos tipos de enemigos
        \begin{itemize}
            \item Tipo básico: 2 vidas
            \item Tipo jefe: 4 vidas
        \end{itemize}
    \item RF-5: Los enemigos aparecen en oleadas y se limitan a avanzar oscilando hacia su objetivo.
    \item RF-6: Cuanto menos se acerquen a la zona inferior de la pantalla mayor será la puntuación que otorgue su eliminación.
    \item RF-7: Si los enemigos llegan a nuestro territorio restarán una cantidad fija de puntos, por lo que nuestra puntuación final se verá notablemente mermada.
    \item RF-8: El juego comienza con la pantalla vacía y el jugador en la parte inferior.
    \item RF-9: Se suma puntos al eliminar enemigos.
    \item RF-10: Cada $0,25$ segundos se debe guardar el estado en forma de instancia en un fichero.
        \begin{itemize}
            \item RF-10.1: La instancia debe tener como mínimo la información de la posicióbn del jugador y la posición de cuatro enemigos. 
        \end{itemize}
\end{itemize}