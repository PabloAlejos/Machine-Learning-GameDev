\capitulo{1}{Introducción}

Históricamente los agentes inteligentes (ya sea un robot físicamente presente formado por un hardware o simplemente un software capaz de hacer web scraping o capaz de jugar al ajedrez), necesitan la definición de rutinas para poder realizar la labor para la que ha sido creados. Esta labor de introducir las instrucciones en su sistema de control recae sobre el programador. Programar estos agentes abarca desde los más sencillos, que realizan una actividad muy concreta, a complejos sistemas capaces de realizar tareas muy sofisticadas. Estos últimos requieren de un gran esfuerzo por parte del programador, pero, ¿y si los agentes se programasen a sí mismos? Gracias a técnicas como Machine Learning podemos hacer que los agentes aprendan por si solos a realizar las tareas para las que fueron creados.
Este proyecto tiene como objetivo final entrenar un agente inteligente capaz de aprender a jugar a un determinado videojuego mediante imitación del ser humano. Esto requiere un gran volumen de datos y una persona haciendo uso de juego para generar los datos de entrenamiento necesarios.