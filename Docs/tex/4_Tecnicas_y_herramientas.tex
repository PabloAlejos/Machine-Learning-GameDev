\capitulo{4}{Técnicas y herramientas}

Para la realización de este proyecto se han empleado las siguientes herramientas:

\Section{Photoshop}
Photoshop es un software profesional de edición fotográfica. El mi proyecto ha sido utilizado para dibujar toda la interfaz gráfica de juego.

\Section{Visual Studio}
Visual Studio  es un IDE de Microsoft que sirve para desarrollar aplicaciones para múltiples plataformas. Este IDE ha sido utilizado para escribir el código en c# correspondiente a la funcionalodad del videojuego.

\Section{Scikit-Learn}
\cite{scikit} Scikit-learn es una libraría de herramientas de Machine Learning. Esta librería nos facilita el entrenamiento de redes neuronales.

\Section{Pandas}
\cite{mckinney-proc-scipy-2010} \cite{mckinney-proc-scipy-2011}Para el manejo de la enorme cantidad de datos obtenidos se ha utilizado Pandas. Esta herramienta nos permite organizar y clasificar grandes cantidades de datos 

\Section{DEAP}
\cite{fortin2012deap} Cosas de DEAP

\Section{Unity}
Como base para el grueso de proyecto se ha utilizado el motor de desarrollo de videojuegos Unity.
Hablar de monobehaiviour.

\Section{GitHub}
GitHub es un herramienta que nos permite llevar un control de versiones de nuestro código fuente, además de permitir el trabajo colaborativo.


