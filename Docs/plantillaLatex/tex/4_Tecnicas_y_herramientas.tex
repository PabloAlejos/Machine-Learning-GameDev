\capitulo{4}{Técnicas y herramientas}

Para la realización de este proyecto se han empleado las siguientes herramientas:

\textbf{Photoshop}
Photoshop es un software profesional de edición fotográfica. El mi proyecto ha sido utilizado para dibujar toda la interfaz gráfica de juego.

\textbf{Visual Studio}
Visual Studio  es un IDE de Microsoft que sirve para desarrollar aplicaciones para múltiples plataformas. Este IDE ha sido utilizado para escribir el código en c# correspondiente a la funcionalodad del videojuego.

\textbf{Scikit-Learn}
Scikit-learn es una libraría de herramientas de Machine Learning. Esta librería nos facilita el entrenamiento de redes neuronales.

\textbf{Pandas}
Para el manejo de la enorme cantidad de datos obtenidos se ha utilizado Pandas. Esta herramienta nos permite organizar y clasificar grandes cantidades de datos 

\textbf{Unity}
Como base para el grueso de proyecto se ha utilizado el motor de desarrollo de videojuegos Unity.

\textbf{TexMaker} 

\textbf{ShareLatex}

\textbf{GitHub}

\textbf{Sublime Text}

