\capitulo{1}{Introducción}

Los robots industriales necesitan la definición de rutinas para poder realizar su la labor para la que ha sido creado. Esta labor de introducir las instrucciones en su sistema de control recae sobre el programador. Los programas introducidos en los robots van desde los más sencillos, que realizan una actividad muy concreta, a complejos sistemas capaces de realizar tareas muy sofisticadas. Estos últimos requieren de un gran esfuerzo por parte del programador, pero, ¿y si los robots se programasen a sí mismos? Gracias a técnicas como Machine Learning podemos hacer que los robots aprendan por si solos a realizar las tareas para las que fueron creados.
Este proyecto tiene como objetivo final entrenar una red neuronal capaz de aprender a jugar a un determinado videojuego mediante imitación del ser humano. Esto requiere un gran volumen de datos y una persona haciendo uso de juego para obtener la información necesaria.